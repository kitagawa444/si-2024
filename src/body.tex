\section{Iterative Closest Pointを利用したSLAM原点の補正}

\begin{figure}[h]
\includegraphics[width=\columnwidth]{figs/2.pdf}
\caption{3D MAP of TOYOTA factory}
\label{fig:fig2}
\end{figure}

 ロボットが工場内における絶対位置を把握するため,Iterative Closest Point\cite{ICP}を利用したSLAM原点の補正アルゴリズムを構築した.
はじめに,3D LiDAR SLAMによりFig.~\ref{fig:fig2}のように環境地図を作成する.事前作成された環境地図と3D LiDARから得られる点群情報にIterative Closest Point (ICP)を適用することにより,地図原点とSLAM原点との相対関係を算出し,補正する.

\section{自律走行システム}
 ロボットの効率的な自律移動および停止時における高精度な位置制御を実現するために,Fig.~\ref{fig:fig3}のような目標位置までの距離によってことなるアプローチをとる自律走行システムを構築した.

目標位置の遠方の場合には大局的および局所的な経路計画を生成することのできる$move\_base$\cite{move_base}を利用した.事前の環境地図は2.節で作成した,3D環境地図を特定の高さ範囲で区切ることで,二次元環境地図を作成した.また,同様に,3D LiDARで得られる点群情報を特定の高さ範囲で区切ることで二次元コストマップをオンラインで作成し,事前マップからの変化や動的な障害物回避に用いた.

目標位置の近傍では,式1で表されるP制御を用いてロボットの位置姿勢制御を行った.二次元環境地図における目標位置及び目標姿勢に追従する.
%% \begin{align}
%% v^{\text{des}} &= K_{P} e_{r} + K_{I_{r}} \int \left e_{r} \right dt  
%% \end{align}
\begin{align}
v^{\text{des}} &= K_{P} e_{r} %%+ K_{I} \int e_{r} \, dt
\end{align}
\begin{align}
w^{\text{des}} &= K_{P} e_{θ} %%+ K_{I} \int e_{r} \, dt
\end{align}

上述の大域的,局所的な対象的な2つの制御を目標位置までの距離を監視し,使い分けることで,目標位置にたいして高精度に追従するシステムを構築した.

\begin{figure}[h]
\includegraphics[width=\columnwidth]{figs/3.pdf}
\caption{Conceptual scheme of autonomous moving system}
\label{fig:fig3}
\end{figure}

\section{システムアーキテクチャ}

本システムの概要をFig.~\ref{fig:fig4}に示す.
本システムの開発にはロボット用のミドルウェアとして広く利用されているROS(Robot Operating System)を利用し,State Estimator, Motion Planner and Position Controllerの処理に活用された.
Onboard PCとして,Khadas社製Vim4(KVIM4-B-001),3D LiDARとしてLivox社製Mid-360,全方位移動台車ロボットとしてヴィンストン社製メカナムローバーver3.0を用いた.

\begin{figure}[h]
\includegraphics[width=\columnwidth]{figs/4.pdf}
\caption{System architecture}
\label{fig:fig4}
\end{figure}

Fig.~\ref{fig:fig5}における台車ロボットには次節で詳説するロボットの軌跡を描くためのペンホルダーを左右に増設した.

\begin{figure}[h]
\includegraphics[width=\columnwidth]{figs/5.pdf}
\caption{Robot platform}
\label{fig:fig5}
\end{figure}

\section{実験}
\subsection{手続き}
 本実験では2,3節で述べた自己位置推定及び,位置制御の正確性を検証した.工場における運用を考え,Fig.~\ref{fig:fig6}のように,始点,終点,中継地点2つの合計4点を目的の姿勢で通過する往復走行をおこなった.\ref{fig:fig6}に示すように,走行車両の両側にはペンを固定し,床面に軌跡を描くことで,得られた軌跡からpoint1,point2におけるロボットの停止精度を分析した.本実験はトヨタ自動車株式会社本社工場で実施された.

\begin{figure}[h]
\includegraphics[width=\columnwidth]{figs/6.pdf}
\caption{Passage of the robot in the experiment}
\label{fig:fig6}
\end{figure}
\begin{figure}[h]
\includegraphics[width=\columnwidth]{figs/7.pdf}
\caption{Experiment and trajectory of the robot}
\label{fig:fig7}
\end{figure}


\subsection{実験結果}
 図2に各停止位置における,左右輪の位置を記載する.繰り返し精度は10~15mm出会ったものの,往復運動で精度に差が現れた.これは回転によってSLAMの精度が減少することが示されている.

%% \begin{figure}[h]
%% \includegraphics[width=\columnwidth]{figs/8.pdf}
%% \caption{ガザニア}
%% \label{fig:fig8}
%% \end{figure}
%% \begin{figure}[h]
%% \includegraphics[width=\columnwidth]{figs/9.pdf}
%% \caption{ガザニア}
%% \label{fig:fig9}
%% \end{figure}
%% \begin{figure}[h]
%% \includegraphics[width=\columnwidth]{figs/10.pdf}
%% \caption{ガザニア}
%% \label{fig:fig10}
%% \end{figure}
