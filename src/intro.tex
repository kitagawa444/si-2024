\section{はじめに}
 現在,トヨタの多くの工場では,効率化のために部品運搬としてAGVやAMRといった自動運搬車両が利用されている.この車両の上に専用の治具を取り付けて,特定の部品や組み立てに必要な工具類を運搬している.本車両には2D LiDARが搭載されており,規定のルートを走行し運搬するためにAMCLやSLAM\cite{2d-AMCL, 2d-SLAM}といったLocalizationやNavigationに活用されている.しかしながら,車両バッテリーの無線充電や部品の受け渡しといった±10mmほどの高い精度を必要とする場面に置いては,停止位置精度が十分でなく,Fig.~\ref{fig:fig1} に示すような磁気テープによるライン検知など複数の技術を組み合わせ,精度を補強することによって実現される.さらに,2D LiDARによる自己位置推定技術は,ライン変更や配置変更などの環境変化が頻繁に起こる工場内のような環境に弱いという欠点がある.

そこで3D LiDARによるSLAM技術が提案されている\cite{FAST-LIO}.3D LiDARによる自己位置推定では2D LiDARと比較して,変動の可能性の低い天井などの高所空間情報を利用できるため,周囲環境の変化が起こったとしても,非常に高い精度で自己位置を推定することができる.しかしながら,SLAM技術は未知環境での自己位置推定に用いられる技術であり,工場のような既知環境での運用には,工場内の座標系に変換するために,起動場所を一定に保つ必要がある.

 そこで本研究では3D LiDAR SLAMの技術に原点補正を加え,3D-MAPを利用した汎用自己位置推定を利用した全方位台車ロボットの高精度自律走行システムを構築し工場内の実環境で検証を行った.
\begin{figure}[htbp]
\begin{flushleft}
\includegraphics[width=\columnwidth]{figs/1.pdf}
\caption{The AGV used in TOYOTA factory}
\label{fig:fig1}
\end{flushleft}
\end{figure}
