\section{はじめに}
 現在,トヨタの多くの工場では,効率化のために部品運搬としてAGVやAMRといった自動運搬車両が利用されている.この車両の上に専用の治具を取り付けて,特定の部品や組み立てに必要な治工具類を運搬している.本車両には2D LiDARが搭載されており,規定のルートを走行し運搬するためにAMCLやSLAM\cite{2d-AMCL, 2d-SLAM}といったLocalizationやNavigationに活用されている.しかしながら,車両バッテリーの無線充電や部品の受け渡しといった±10mmほどの高い精度を必要とする場面に置いては,停止位置精度が十分でなく磁気テープによるライン検知など複数の技術を組み合わせ,精度を補強することによって実現される.さらに,2D LiDARによる自己位置推定技術は,ライン変更や配置変更などの環境変化が頻繁に起こる工場内のような環境に弱いという欠点がある.

そこで3D LiDARによるSLAM技術が提案されている\cite{6DSLAM_outside, 3D_SLAM_outside, EKF_SLAM, FAST-LIO, FAST-LIO2}.3D LiDARによる自己位置推定では2D LiDARと比較して,変動の可能性の低い天井などの高所空間情報を利用できるため,周囲環境の変化が起こったとしても,非常に高い精度で自己位置を推定することができる.しかしながら,SLAM技術は未知環境での自己位置推定に用いられる技術であり,工場のような既知環境での運用には,工場内の座標系に変換するために,起動場所を一定に保つ必要がある.

そこで本研究ではこのロバストな3D LiDARによるSLAM技術を既知環境での繰り返し動作で活用するため,三次元点群情報を利用した自己位置推定技術へ拡張した.また,部品運搬時のロボットの効率的な自律移動と高精度な目標位置・姿勢への追従を実現する,高精度自律走行システムを開発した.これらの技術を統合し,実際の工場におけるシステム運用とロボット停止精度の検証実験を行った.本稿では,3次元点群情報を利用した自己位置推定技術,台車の経路計画・追従制御手法,システムアーキテクチャ,トヨタ自動車株式会社本社工場で行った実環境工場検証実験について記す.

\begin{figure}[!t]
\includegraphics[width=\columnwidth]{figs/2.pdf}
\caption{Omini-directional robot platform wth mecanum wheel and 3D LiDAR. Result of ICP in TOYOTA factory}
\label{fig:fig2}
\end{figure}
